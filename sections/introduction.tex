% % % % % % % % % % % % % % % % % % % % % % % % % % % % % % % % % % % % % % % 
% % % % % % % % % % % % % % % % % % % % % % % % % % % % % % % % % % % % % % % 
% % %   Originally created by GC for template in EPFL MSc/BSc theses.   % % %
% % %                           February 2022                           % % %
% % % % % % % % % % % % % % % % % % % % % % % % % % % % % % % % % % % % % % % 
% % % % % % % % % % % % % % % % % % % % % % % % % % % % % % % % % % % % % % % 

\section{Introduction}
\label{sec:thesis_introduction}  % % feel free to change the label name to use a more descriptive name. For instance, 'sec:poly_introduction' if your thesis has to do with polynomials.

In this section, you introduce the reader into the thesis. You should describe the background context, and slowly motivate your work.
The next few sections can be organized in a different way depending on the type of thesis.
You can add citations, e.g., \cite{goodfellow2014generative}, or Tables, such as Table~\ref{tab:title_please_change_me}. 

 \begin{table}[!h]
    \caption{This is a sample table.}
    \centering
     \begin{tabular}{|c |c| c|c|} 
       This & and & That & table\\
     \end{tabular}
     \label{tab:title_please_change_me}
\end{table}
